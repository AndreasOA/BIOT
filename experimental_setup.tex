\section{Experimental Setup and Analysis}

\subsection{Model Architecture}
We conducted experiments using the BIOT (Biosignal Transformer) architecture \cite{yang2023biot} as our base model. The key modification in our study was replacing the Linear Transformer component of BIOT with different variants of the xLSTM architecture \cite{beck:24xlstm}. Specifically, we implemented the following model configurations:

\begin{itemize}
    \item \textbf{Linear Transformer}: The original BIOT architecture
    \item \textbf{M+S-LSTM}: Combination of M-LSTM and S-LSTM components
    \item \textbf{M-LSTM}: Using only the M-LSTM variant
    \item \textbf{S-LSTM}: Using only the S-LSTM variant
\end{itemize}

\subsection{Data Processing and Sample Length}
For each experiment, we processed the biosignal data with different sample lengths (5, 7, and 9 seconds). The sample window is always centered on the event of interest, with the following distribution:
\begin{itemize}
    \item 5-second window: 2 seconds before + 1 second event + 2 seconds after
    \item 7-second window: 3 seconds before + 1 second event + 3 seconds after
    \item 9-second window: 4 seconds before + 1 second event + 4 seconds after
\end{itemize}

\subsection{Evaluation Metrics}
We evaluated our models using both primary and secondary metrics:

\subsubsection{Primary Metrics}
The following metrics were used as the main evaluation criteria, consistent with the BIOT paper:
\begin{itemize}
    \item Balanced Accuracy
    \item Cohen's Kappa
    \item F1 Score
\end{itemize}

\subsubsection{Secondary Metrics}
Additional metrics were included for comprehensive comparison and analysis of model performance.

\subsection{Experimental Protocol}
For each model configuration and sample length, we conducted three independent runs using the default settings from the BIOT paper. This approach ensures:
\begin{itemize}
    \item Statistical reliability of the results
    \item Consistency with the original BIOT implementation
    \item Fair comparison between different architectural variants
\end{itemize}

\subsection{Analysis Methods}
We performed two types of analysis on our experimental results:

\subsubsection{Comprehensive Analysis}
The first analysis provides a detailed view of model performance, including:
\begin{itemize}
    \item Maximum values achieved during training
    \item Mean performance across runs
    \item Variance in performance
\end{itemize}
This analysis is particularly useful for understanding the full range of model capabilities and stability.

\subsubsection{Simplified Analysis}
The second analysis focuses on the mean performance with variance, providing a more concise view of:
\begin{itemize}
    \item Average performance across runs
    \item Consistency of results (through variance)
\end{itemize}
This simplified view helps in quickly comparing the overall effectiveness of different model configurations.

\bibliographystyle{plain}
\begin{thebibliography}{2}

\bibitem{yang2023biot}
Yang, C., Westover, M. B., \& Sun, J. (2023).
\newblock BIOT: Biosignal Transformer for Cross-data Learning in the Wild.
\newblock In {\em Thirty-seventh Conference on Neural Information Processing Systems}.
\newblock \url{https://openreview.net/forum?id=c2LZyTyddi}

\bibitem{beck:24xlstm}
Beck, M., Pöppel, K., Spanring, M., Auer, A., Prudnikova, O., Kopp, M., Klambauer, G., Brandstetter, J., \& Hochreiter, S. (2024).
\newblock xLSTM: Extended Long Short-Term Memory.
\newblock In {\em Thirty-eighth Conference on Neural Information Processing Systems}.
\newblock \url{https://arxiv.org/abs/2405.04517}

\end{thebibliography} 